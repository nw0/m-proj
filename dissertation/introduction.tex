\documentclass[dissertation.tex]{subfile}

\chapter{Introduction}
\label{ch:intro}

\section{Motivation}
\label{sec:intro-motiv}

\wip{This is chopped up from the proposal}
\subsection{Background}
Many security vulnerabilities are bugs arising from the lack of memory safety in C; an example of a traditional exploit is the buffer overflow.
The C language specification includes a number of situations that lead to \emph{undefined behaviour}: akin to a deduction of `falsity' in logic, after these points the state of the program might be arbitrary.
Undefined behaviour allows compilers to assume that certain conditions never hold, enabling them to make powerful (but dangerous) optimisations.

Consequently, crucial checks can be optimised away and invariants violated by compilers, resulting in executables vulnerable to memory attacks~\cite{simon2018wygiwyc}.
Many systems and compiler programmers frequently underestimate the pervasiveness of undefined behaviour in code that appears to function correctly, and gloss over the subtleties of the C standard~\cite{memarian2016cdepths};
these represent future opportunities for vulnerabilities to arise when code is optimised away by compiler transformations~\cite{wang2013towards}.

\subsection{Capabilities and the CHERI architecture}
The CHERI project provides \emph{capabilities}, which stops unauthorised access or modification of data by erroneous and malicious code~\cite{cheri-v6}.
In the CHERI architecture, capabilities restrict usage by associating bounds and access permissions with address data while making these unforgeable, as they can only be manipulated by the hardware.
This prevents abuse of pointers and internal addresses.


\section{Intended contributions}
\label{sec:intro-contrib}
