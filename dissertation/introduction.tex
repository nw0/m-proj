\documentclass[dissertation.tex]{subfiles}
\begin{document}

\chapter{Introduction}
\label{ch:intro}

\section{Motivation}
\label{sec:intro-motiv}

\subsection{Background}
Many security vulnerabilities are bugs arising from the lack of memory
safety in C; an example of a traditional exploit is the buffer overflow.
The C language specification includes a number of situations that lead
to \emph{undefined behaviour}: akin to a deduction of `falsity' in
logic, after these points the state of the program might be arbitrary.
Undefined behaviour allows compilers to assume that certain conditions
never hold, enabling powerful (but dangerous) optimisations.

Consequently, crucial checks can be optimised away and invariants
violated by compilers, resulting in executables vulnerable to memory
attacks~\cite{simon2018wygiwyc}.
Many systems and compiler programmers frequently underestimate the
pervasiveness of undefined behaviour in code that appears to function
correctly, and gloss over the subtleties of the C
standard~\cite{memarian2016cdepths}.
These represent opportunities for vulnerabilities to arise when code is
optimised away by compiler transformations~\cite{wang2013towards}.


\subsection{The Rust programming language}

Rust is billed as a fast, safe language designed for concurrent systems
programming.
Safe Rust claims to offer type- and memory-safety, and provides
guarantees that no dangling pointers or any undefined behaviour will
occur~\cite{rust-nomicon-safe-unsafe}, giving both spatial and temporal
memory protection.

To achieve this, it uses compile-time and runtime checks, relying on a
combination of LLVM optimisations and \emph{Unsafe} Rust code to
minimise the performance overhead.
However, as with all code that requires fine control over memory,
mistakes in writing Unsafe Rust give rise to potential vulnerabilities.

\subsubsection{Vulnerabilities in Rust}
There have been several unrelated bugs and potential vulnerabilities in
Rust, all believed to be unexploited.
Some are preventable with CHERI capabilities, and these are examined in
\Cref{sec:eval-bugs}:

\begin{itemize}
    \item Standard library: buffer overflow when pushing to a
    \texttt{VecDeque}, CVE-2018-1000657~\cite{cve-push,rust-issue-push}.
    \item Standard library: buffer overflow in slice \texttt{repeat},
    CVE-2018-1000810~\cite{cve-repeat,rust-advisory-repeat,rust-pr-slice}.
    \item Standard library: out-of-bounds access indexing into a
    reversed slice~\cite{rust-pr-reverse,rust-commit-reverse}.
    \item Standard library: unsafe Iterator method duplicates
    exclusive mutable references, violating temporal
    safety~\cite{rust-issue-vec-mut}.
    \item Standard library: safe trait implementation allows arbitrary
    typecasting and thus buffer overflows,
    CVE-2019-12083~\cite{cve-cast,rust-advisory-cast}.
\end{itemize}

Others are not mitigated by capabilities:

\begin{itemize}
    \item Standard library: attempt to read uninitialised memory after
    appending to a \texttt{VecDeque}~\cite{rust-issue-deque-append}.
    \item Compiler plugin: documentation plugin allowed arbitrary code
    execution by a different user while running the
    compiler, CVE-2018-1000622~\cite{cve-rustdoc,rust-advisory-rustdoc}.
\end{itemize}

Rust is a language with an explicit priority of memory safety, developed
by a community of safety- and security-conscious programmers.
Collectively, these flaws demonstrate that such favourable conditions
are not sufficient to eliminate unsoundness and vulnerabilities.
On the contrary, more tools are needed to build and operate reliable and
secure software.


\subsection{Capabilities and the CHERI architecture}
The CHERI instruction set architecture replaces pointers with
\emph{capabilities}, which stop unauthorised access of memory as they are
unforgeable in software~\cite{cheri-v6}.
Capabilities protect data by associating bounds information and access
permissions, giving programs fine-grained mechanisms to enforce the
security principles of \emph{least privilege} and \emph{intentional
use}~\cite{neumann-principles}.

Just as using capabilities in C programs prevents unintended operations
on memory, applying capabilities to Rust provides an additional layer of
safety, guarding against undiscovered vulnerabilities.


\section{Contributions}
\label{sec:intro-contrib}

Traditionally, capability mechanisms have been evaluated against C, but
the main objective of this project was to evaluate the utility of
porting Rust, a safe language, to CHERI.
In summary, the contributions of this work are:

\begin{itemize}
    \item A summary of features and techniques Rust uses to provide
    memory safety guarantees, and how it optimises these.
    The differences between Rust and other languages, as they pertain to
    capability architectures and CHERI.
    A survey of work relating to safe languages and capability
    platforms, and efforts to manage undefined behaviour and safety in
    programming languages.

    \item Patches to the Rust compiler (1.35) and core libraries which
    enable compilation to the new \texttt{cheri-unknown-freebsd} target
    utilising 128-bit CHERI capabilities.
    These compile programs which use the Rust core library without
    optimisations.

    \item An analysis of previous errors leading to vulnerabilities in
    Rust, the memory safety implications, and how they are mitigated
    using capabilities.
    Two demonstrative microbenchmarks of vulnerabilities in the Rust
    standard library, shown to be functional on x86 but prevented by
    CHERI capabilities.

    \item An evaluation of the interactions between Rust and CHERI
    protection mechanisms, and how hardware capabilities enhance Rust in
    general.
    An exploration of how features provided by the CHERI architecture
    can be used to enforce Rust guarantees where its compiler and
    runtime cannot, even as programs call into untrusted code and the
    kernel.

    \item Techniques and approaches that may be used to reduce the
    overheads posed by memory protection when combining the two
    approaches.
    A consideration of how the duplication of functions can be
    minimised, and how doing so affects the provided safety guarantees.

    \item Details of the relevant concerns for future implementers
    porting Rust to CHERI, or examining CHERI on other safe languages.
    This includes a discussion on problematic points of the Rust
    language which may need to be clarified, changed, or implemented
    differently.
\end{itemize}

This work is \emph{not} intended to be a complete implementation of
CHERI support in the Rust compiler.
Rather, it prototypes support for replacing pointers with capabilities
in Rust, to illustrate the benefit of CHERI capabilities to a safe
language.


\section{Report structure}
\todo{}

\end{document}
