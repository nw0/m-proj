\documentclass[dissertation.tex]{subfiles}
\begin{document}

\chapter{Conclusion}

\section{Context and review}

This project examined protection mechanisms to provide memory safety at
the intersection of hardware and computer architecture, compilers, and
programming language semantics.
The varied approaches and measures shed light on the subtlety of
two orthogonal problems: spatial and temporal integrity.
These insights illustrate different ways of enforcing the principles of
\emph{least privilege} and \emph{intentional use}, and the compromises
and difficulties that arise from their interaction.

\todo{Comments on the work. What was special or challenging?}

Even as CHERI begins to form a foundation for its own temporal memory
protection, the study of how the architecture interacts with memory-safe
languages forms a basis for realising its potential in the larger
ecosystem of modern software development.


\section{Scope of contributions}

\section{Further work}

\subsection{Clarify Rust's provenance model}

\subsection{Pointer-width sized indices}
\todo{These aren't necessarily helpful: propose to use `address-width' usize through rfc}

\subsection{Resolution of \texttt{usize} and pointer width in the compiler}
\todo{Also, enable optimisations}

\subsection{Runtime and memory overhead analysis}
\todo{Can use miri to count pointers in the heap? Hence do memory analysis}

\subsection{Fine-grained capability protection}

\end{document}
