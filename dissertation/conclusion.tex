\documentclass[dissertation.tex]{subfiles}
\begin{document}

\chapter{Conclusion}

\section{Context and review}

This project examined memory protection mechanisms at
the intersection of hardware and computer architecture, compilers, and
programming language semantics.
The varied approaches and measures shed light on the subtleties of
two orthogonal problems: spatial and temporal integrity.
These insights illustrate different ways of enforcing the principles of
\emph{least privilege} and \emph{intentional use}, and the compromises
and difficulties that arise from their interaction.

Whereas the benefits of hardware capabilities are well-studied for C,
this work provides an understanding of their benefits in the context of
a safe language, Rust.
Each safe language provides different guarantees and enforces them
differently; Rust defines semantics and performs most of its enforcement
through static checking, rejecting programs with unclear invariants.
The strict constraints that this places on programs motivates Unsafe
Rust, which provides programmers more control over execution.
In the standard library, unsafe code is even used to optimise string
processing, a traditional source of C buffer overflows.
\Cref{sec:rust-unsafe,sec:rust-elision} covered situations in which
Unsafe Rust is used.

It is in unsafe code that capabilities demonstrate their value,
providing dynamic checks where static checking is not possible.
\Cref{sec:eval-bugs} examined how Rust's protections yield to
programmer error, giving analyses of previous vulnerabilities in the
Rust standard library.
Further, it showed that capabilities successfully prevent these
vulnerabilities from exploitation.
Indeed, in the course of writing this report, another vulnerability in
the Rust standard library surfaced~\cite{rust-advisory-cast}: it too
is preventable by CHERI capabilities!

The evaluation went on to discuss how capabilities can strengthen and
enforce guarantees given by the language, using features like
\emph{capability sealing} to protect data given to cross-domain calls.
This protection obviates expensive measures like sandboxing, frequently
considered too costly to use in many real-world applications.
I emphasise that capabilities equip Rust with powerful
mechanisms for increasing memory safety at low cost,
\emph{without} changing the language semantics or memory model.
A similar claim cannot be made for C, due to its weak pointer
provenance model~\cite{cheri-2019-abstract}.
Further, I elaborate on how capabilities interact with, and can rule out
several forms of undefined behaviour in Rust.
Compiler transformations relying on those forms are thus safe to
perform on CHERI, without concern of violating the integrity or
confidentiality of data.

I also elucidated how Rust's techniques complement CHERI memory safety,
offering a clear pointer provenance model, a safe sub-language, and
complementary temporal safety guarantees.
These properties make Rust code easier to compile with capabilities,
enable the targeted use of capabilities to protect the most vulnerable
data, and avoid the most common vulnerabilities that
CHERI architectures do not currently address.
Combined, these properties permit optimisations on capability usage not
possible with C codebases, showcasing the promise of combining
capability architectures with safe languages.
With Rust performance comparable to that of C without capabilities,
these optimisations could mean that Rust code will run even \emph{faster}
than C code under capability protection.
This is especially significant as workloads become more bounded by memory
throughput, where capability overheads are most noticeable.

Even as CHERI begins to form a foundation for its own temporal memory
protection, the study of how the architecture interacts with memory-safe
languages forms a basis for realising its potential in the larger
ecosystem of modern software development.
As a preliminary step, \Cref{sec:eval-porting} covers the initial
concerns when porting safe languages to capability architectures.


\section{Challenges}

This project has not been without its challenging points.
A recurring problem was the conflict caused by changing the pointer
width to 128 bits, various aspects of which were recorded in
\Cref{sec:rust-usize,sec:impl-width,sec:eval-usize}.
The issue effectively prevents full Rust compatibility with CHERI
capabilities, and in this project prevents the use of compiler
optimisations.
The definition is part of the language specification, and a source of
some controversy in the Rust community; I put forth small, plausible
changes to remove the conflict with CHERI.
A detriment of this issue has been the inability to perform more
sophisticated evaluation of CHERI overheads in Rust, as Joannou et al.\
have done for C~\cite{cheri-tagged-memory}.

Another challenge was maintaining the relevance of the work.
At first glance, the motivation is obvious.
Capabilities make C much safer: can they do the same for a safe language
like Rust?
Could they optimise out dynamic checks like array bounds checks, leading
to a measurable performance increase?

The nuances emerge when one considers Unsafe Rust, or the diverse
`escape hatches' in other languages.
Perhaps the relevant questions are those of where languages benefit the
most, or equally where capabilities are \emph{least} useful.
The dearth of literature studying capabilities under safe programming
languages does not help;
indeed the proposition that capabilities are less obviously useful for
safe languages seems increasingly reasonable.
These serve to illustrate the value of segmenting safe and unsafe code
and enabling fine-grained protection in the language
(\Cref{sec:eval-capability-using-type,sec:eval-hybrid}).


\section{Scope of contributions}

This project was principally concerned with the challenges that arise
from bringing a safe language to the CHERI architecture, and how they
interact with each other.
In providing an account of these concerns, it made a number of
contributions:

\clar{I omit the first bulleted contribution from the introduction, to
do with the background and related work chapters.

I am planning to delete said point from that section. Should I do so?}

\begin{itemize}
    \item Patches to the Rust compiler and core libraries enabling
    support for CHERI capabilities, without compiler optimisations.
    Details of the necessary further work to fully support a CHERI
    target, including the language design choices which pose problems
    for this task.
    These modifications formed the basis on which I evaluated Rust on
    CHERI.

    \item Verification that the modified compiler produces programs
    which use capabilities, by demonstrating capability protection
    against previous vulnerabilities.
    Analyses of these and other vulnerabilities, which were all present
    in the Rust core library, showing that capabilities are effective in
    fully mitigating the errors.

    \item An evaluation of the interactions between Rust and CHERI
    protection mechanisms, focussed on how the two complement each
    other to provide stronger memory safety guarantees.
    I elucidated how CHERI provisions such as \emph{capability sealing}
    bring new memory safety properties to Rust, especially the means
    to enforce previously-unenforceable invariants on which Rust
    predicates its memory safety guarantees.

    \item The evaluation further investigated approaches that reduce the
    overheads posed by these memory protection mechanisms without
    reducing memory safety.
    It proposed configurations that avoid unnecessary duplication of
    strong protections, but instead work to reinforce the most
    vulnerable points of Rust programs: the use of unsafe code and
    cross-domain calls.
    Further, I highlighted the trade-offs of different combinations of
    protection techniques and how they affect memory safety, crucial
    factors for implementers bringing Rust to CHERI.

    \item I addressed details of the Rust language and implementation
    that currently prevent full Rust compatibility with CHERI.
    Of specific interest is the definition of \texttt{usize} in the
    language, changes to which would need to be put to the Rust
    community.
    Also discussed is how different aspects of protection and
    implementation might apply to other safe languages and their
    runtimes, providing insight into CHERI's implications for the wider
    software development ecosystem.
\end{itemize}

\section{Further work}

\subsection{Pointer-width sized indices}
\label{sec:further-usize}

\subsubsection{Changes to Rust semantics}
\Cref{sec:eval-usize} considered Rust's \texttt{usize} definition, which
is ``pointer-sized integer'', why this is not desirable, and explored
alternatives.
As this is both a semantic and performance issue on CHERI
(\Cref{sec:impl-width}), it should be a priority to resolve this
definition sooner rather than later.

Nevertheless, this is not a straightforward task.
A unilateral change would involve large divergences from the upstream
code, as the pointer width and index sizes are implicitly used
interchangeably in the Rust compiler.
Since such a change affects the Rust semantics, and potentially
introduces new types, it inevitably will have to be proposed via the
Rust RFC system.
The issue has been discussed before~\cite{rust-rfc-isize,rust-int-uint};
I suggest that the least controversial change would be to reinterpret
the current rule and leave the introduction of any new types for a later
effort.
This could take a significant amount of time and human effort to effect,
and is entirely beyond the scope of this exploratory work.

\subsubsection{Implementation in the compiler}
This change could then be implemented in the compiler.
As noted above, this would be challenging to do without first achieving
consensus on the semantics.

While there are numerous references that do not fully distinguish
between pointer size and index size, the CTSRD project has experience
resolving similar issues in Clang and LLVM, showing that the problem is
not insurmountable.


\subsection{Runtime and memory overhead analysis}
This work demonstrated that Rust protections can work alongside CHERI
capabilities and explored the safety and semantic implications.
However, the runtime and memory overheads of CHERI have not been
explored.

\subsubsection{Approaches}
To measure runtime overheads, a possibility is to compare against a MIPS
FreeBSD target, which I include in the patches provided in this project.
These should compile any Rust program with optimisations, although it
should be noted that Rust releases are not tested against the MIPS64
Linux and amd64 FreeBSD targets from which this
derives~\cite{rust-platform-support}.
The major reason this comparison is not currently possible is the lack
of optimisations, and performance issues caused by 128-bit index sizes
(see \Cref{sec:further-usize}), both beyond the scope of this work.

For memory overheads, similar techniques to those employed by Joannou et
al.\ \cite{cheri-tagged-memory} can be used, by core-dumping Rust
programs.
Another plausible approach might be to use Rust's Mid-level IR
interpreter (\emph{Miri}) to track heap pointers, and compute overheads
from there.
This would give precise measurement of overheads, as well as present an
effective memory analysis tool for Rust code running on any platform.


\subsection{Fine-grained capability protection}

Previous Rust bugs resulting in memory vulnerabilities occur in unsafe
code.
This is because Safe Rust does not have the expressivity to dereference
memory except in specific ways, including arrays and structs.
Further, the Rust developers are keen to uphold the tenet that Safe Rust
should never exhibit undefined behaviour, and on multiple occasions have
removed or marked unsafe code which violated this assumption.
They do so on the grounds that this is incorrect behaviour.

It is therefore reasonable to assume that this will continue, and Rust
vulnerabilities relating to memory protection will only occur through
the use of Unsafe Rust.

\subsubsection{Approaches}
\Cref{sec:eval-capability-using-type} suggested how certain Rust types
requiring particularly sensitive handling could be protected by
capabilities, and \Cref{sec:eval-hybrid} examined the implications of
protecting objects reachable by unsafe code.
In both cases, using regular pointers for the remaining objects could
reduce the memory overheads of capabilities on Rust programs.

Both would be challenging to implement, as Rust would first need to be
extended to support multiple LLVM address spaces, then additional
provenance analysis would be needed to determine which objects should be
protected.

However, the benefits are sizeable if non-capability pointers could be
used widely: this would result in lower memory overheads than C
programs, for instance.
This is also safer than focussing capability protection on risky C
libraries,\footnote{Such as image and video codecs} as provenance is far
clearer in Rust, obviating the possibility of other unchecked code
interfering with the capability protection.

\end{document}
