\documentclass[dissertation.tex]{subfiles}
\begin{document}

\chapter{Conclusion}

\section{Context and review}

This project examined protection mechanisms to provide memory safety at
the intersection of hardware and computer architecture, compilers, and
programming language semantics.
The varied approaches and measures shed light on the subtlety of
two orthogonal problems: spatial and temporal integrity.
These insights illustrate different ways of enforcing the principles of
\emph{least privilege} and \emph{intentional use}, and the compromises
and difficulties that arise from their interaction.

\todo{Comments on the work. What was special or challenging?}

Even as CHERI begins to form a foundation for its own temporal memory
protection, the study of how the architecture interacts with memory-safe
languages forms a basis for realising its potential in the larger
ecosystem of modern software development.


\section{Scope of contributions}

A summary of the contributions of this project:

\begin{itemize}
    \item A summary of features and techniques Rust uses to provide
    memory safety guarantees, and how it optimises these.
    The differences between Rust and other languages, as they pertain to
    capability architectures and CHERI.
    A survey of work relating to safe languages and capability
    platforms, and efforts to manage undefined behaviour and safety in
    programming languages.

    \item Patches to the Rust compiler (1.35) and core libraries which
    enable compilation to the new \texttt{cheri-unknown-freebsd} target
    utilising 128-bit CHERI capabilities.
    These compile programs which use the Rust core library without
    optimisations.

    \item An analysis of previous errors leading to vulnerabilities in
    Rust, the memory safety implications, and how they are mitigated
    using capabilities.
    Two demonstrative microbenchmarks of vulnerabilities in the Rust
    standard library, shown to be functional on x86 but prevented by
    CHERI capabilities.

    \item An evaluation of the interactions between Rust and CHERI
    protection mechanisms, and how hardware capabilities enhance Rust in
    general.
    An exploration of how features provided by the CHERI architecture
    can be used to enforce Rust guarantees where its compiler and
    runtime cannot, even as programs call into untrusted code and the
    kernel.

    \item Techniques and approaches that may be used to reduce the
    overheads posed by memory protection when combining the two
    approaches.
    A consideration of how the duplication of functions can be
    minimised, and how doing so affects the provided safety guarantees.

    \item Details of the relevant concerns for future implementers
    porting Rust to CHERI, or examining CHERI on other safe languages.
    This includes a discussion on problematic points of the Rust
    language which may need to be clarified, changed, or implemented
    differently.
\end{itemize}

\section{Further work}

\subsection{Clarify Rust's provenance model}

\subsection{Pointer-width sized indices}
\todo{These aren't necessarily helpful: propose to use `address-width' usize through rfc}

\subsection{Resolution of \texttt{usize} and pointer width in the compiler}
\todo{Also, enable optimisations}

\subsection{Runtime and memory overhead analysis}
\todo{Can use miri to count pointers in the heap? Hence do memory analysis}

\subsection{Fine-grained capability protection}

\end{document}
