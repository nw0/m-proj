\documentclass[dissertation.tex]{subfiles}
\begin{document}

\chapter{Compiling Rust for CHERI}
\label{ch:impl}


This chapter discusses key details in compiling Rust for CHERI, as
opposed to other architectures.
It details the strategy taken, thereby defining the scope of work in
this project.
To this end, I observe significant architectural differences that make
compilation challenging, and consider design choices for this and future
work porting Rust to CHERI.


\section{Strategy}
\label{sec:impl-what}

\subsection{Host platform}
One contribution of this project is an extension of the Rust compiler to
target FreeBSD on CHERI128.
This is achieved by cross-compilation from an amd64 FreeBSD host; other
host platforms should work if they are able to compile the CHERI SDK.

\subsection{Rust functionality on CHERI}
This extension exclusively targets \emph{pure capability mode}, where
all pointers are capabilities~\cite{cheri-prog-guide}.
This avoids the hurdle of dealing with multiple LLVM address spaces and
pointer widths, not supported in the Rust compiler
(\Cref{sec:impl-addrspace}).

In order to evaluate Rust on CHERI, I identified
a minimal necessary subset of the standard library, listed below.
Not included is the bulk of the Rust standard library, which provides
filesystem and network support, general IO, interfaces to operating
system synchronisation and threading, and others.
I consider these not to have critical significance to capabilities in
standalone Rust programs.

\begin{description}
    \item[\texttt{libcore}] for key definitions of traits and types,
    typechecking, compiler intrinsics, and other fundamental operations.
    It also includes support for calling through foreign function
    interfaces (FFI), Unicode and number formatting, etc.
    It could be thought of as a modern version of the standard C
    headers, apart from the core definitions it provides, which would be
    built into a compiler.
    \item[\texttt{liballoc}] for heap allocation and built-in data
    structures and operations on them.
    \texttt{liballoc} also included two (patched) vulnerabilities in the
    Rust standard library, demonstrated in the evaluation
    (\Cref{sec:eval-micro-push,sec:eval-micro-repeat}).
\end{description}

Other parts of the standard library will compile for CHERI, but they are
not as central to running Rust programs.
I have not included them here in order to focus on evaluating the Rust
compiler and a core set of features.

\Cref{sec:impl-div-compiler,sec:impl-div-lib} detail modifications to
the compiler and these libraries required to compile Rust programs.

\subsection{Compiler optimisations}
Optimisations are \emph{not} used.
Optimisations are not supported at this point as they cause the Rust
compiler to invoke LLVM APIs with values not supported by the CHERI
backend, such as offsets of 128 bits.
\Cref{sec:impl-width} discusses related issues.


\section{The Rust compiler and LLVM}
\label{sec:impl-rustllvm}

Rust uses LLVM as a backend for compilation.
The CTSRD project maintains a fork of LLVM with support for the
current CHERI implementation, extending the MIPS backend to
support CHERI targets, and generic code to handle
capabilities~\cite{cheri-prog-guide}.

\subsection{Incompatibilities with the LLVM backend}
Due to the pointer width and index size divergence
(\Cref{sec:rust-usize}), two incompatibilities were found with the CHERI
LLVM backend.
These have been fixed in the current version.

\begin{itemize}
    \item Rust may attempt to generate an unusual pointer offset when
    calling or returning from functions.\footnote{Fix accessible at
    \url{https://github.com/CTSRD-CHERI/llvm-project/commit/39cfdf711a759a4799bd32f50af07f5f6f43c987};
    thanks to Alexander Richardson.}
    \item Rust sometimes attempts to generate unaligned loads.
    This is not possible in the pure capability ABI.\footnote{Fix
    accessible at
    \url{https://github.com/CTSRD-CHERI/llvm-project/commit/d8e6acf3d4094270cd55b70341a6f9c4d032db81};
    thanks to Alexander Richardson.}
\end{itemize}

Both of these were identified as a result of extensive debugging and
code generation failures, requiring analysis across both the Rust compiler
and LLVM.

The Rust project also maintains a fork of LLVM, with minor changes, and
provide `known good' combinations of LLVM and Rust compiler commits.
This is a good starting point to find a CHERI-supporting LLVM commit
which will compile the Rust compiler.


\subsection{Compiler usage}
Enabling the \cuf target when building the Rust compiler necessitates
compiling the complete Rust standard library for CHERI, which is not
possible at this stage.
The default package manager, Cargo, does not support this situation and
cannot be used.
I identified and used two workflows for compiling programs:

\subsubsection{Integrated compilation with Xargo}
Xargo describes itself as a ``sysroot manager'', meaning that it will
(cross-)compile versions of the libraries for target platforms.
This is useful where binary releases do not exist for those platforms;
it allows the use of modified libraries, and in particular it does not
require that the entire standard library is compiled.

By default it only compiles \texttt{core}; when using \texttt{alloc}
this must be specified through \texttt{Xargo.toml}.
This approach is recommended, although it can make code generation
errors difficult to debug.

Note that Xargo \emph{does not expect dependencies (e.g.\
\texttt{libcore}) to change!}
Therefore, it does not always recompile a ``sysroot'' after
modifications.
Invoking it with a build tool might be advisable.

\subsubsection{Scripted compilation with \texttt{llc} and Clang}
This is the traditional compilation-and-linking process.

\begin{enumerate}
    \item First, libraries are compiled as Rust libraries
    (\texttt{.rlib}) and dynamic libraries (\texttt{.so}).
    This is done by using \texttt{rustc}, not Cargo.
    \item Then, Rust programs are compiled to LLVM IR (again with
    \texttt{rustc}), manually specifying the Rust libraries for
    definitions.
    One can compile to object code directly, but this step can be
    helpful for debugging and ensuring that all the desired target
    attributes and flags are passed in the next step.
    \item Next, \texttt{llc} is used to compile the LLVM IR to object
    files.
    \item Finally, Clang links against the CheriBSD sysroot.
\end{enumerate}

I provide sample Makefiles which use this process for the test programs
in this project, including appropriate command-line arguments.


\section{Changes to the compiler}
\label{sec:impl-div-compiler}

The \cuf target has several differences to most existing targets in the
Rust compiler.
This section documents these differences, their implications, and the
modifications required to compile Rust for CHERI.


\subsection{Pointer width of 128 bits}
\label{sec:impl-width}

\subsubsection{Differences}
Under pure capability mode in CHERI128, all pointers are 128-bit wide
capabilities.
The Rust compiler has built-in support for 16-, 32-, and 64-bit
pointers, through compile-time macros.
The pointer width is used to determine the \texttt{usize}\footnote{The
maximum index size.} (\Cref{sec:rust-usize}), offsets, and sizes used
for pointer operations.

\subsubsection{Choices and implications}
The 128-bit pointer width cannot be changed.

The Rust compiler has two definitions of the target's pointer width: the
\texttt{target\_pointer\_width value} in the target specification
(\Cref{sec:impl-cheritarget}), and the LLVM data layout string.
These values \emph{must} agree to avoid alignment issues.

In any case, the value derived from data layout string is used in more
than 50 locations in the compiler, to determine the index size and
pointer alignment, and for code generation among others.
This includes which index size to use when calling LLVM intrinsics, such
as \texttt{memset}, and what size integer to provide for an
\texttt{inttoptr} call; CHERI only supports 64-bit integers for all
these cases.

The immediately obvious solution is to support a 64-bit index size, but
this is met with a non-functional compiler.
As \Cref{sec:rust-usize} noted, the Rust language defines the
\texttt{usize} to be equal to the pointer width.
This assumption is sufficiently widespread in the compiler to be
impractical to change within the scope of this project.
However, if this definition is changed, or the definition of
\texttt{usize} for typechecking compiled programs can be distinguished
from the usage of index sizes for code generation, then the changes in
the next section may not be necessary.


\subsubsection{Changes and limitations}
This meant defining a \texttt{usize}, and exposing a 128-bit integer
type from LLVM as suitable for this value.
This is not a functional problem: while 128-bit integer usage is not
natively supported on CHERI, it is merely slower and LLVM will generate
the correct instructions.
However, the overhead is significant, as seen in
\Cref{lst:impl-usize128}.

\begin{figure}[ht]
    \lstinputlisting[language={[mips64]Assembler}]{listings/boundscheck-usize128.s}
    \lstinputlisting[language={[mips64]Assembler}]{listings/boundscheck-usize64.s}
    \caption{
        Top: bounds check with 128-bit \texttt{usize} under CHERI;
        bottom: bounds check with 64-bit \texttt{usize} under MIPS64.
        Bounds comparison code when calling into functions which index
        into a given slice; all code before and after checks omitted.
    }
    \label{lst:impl-usize128}
\end{figure}

The other change is to truncate or extend integer types before calls to
LLVM intrinsics and pointer operations.
An example of the former is \texttt{memcpy}: if indexed by a 128-bit
length, LLVM silently omits the copy, as it is not defined in the
CHERI backend.
This change is problematic, as it is possible to miss out the truncation
or extension for some intrinsics, then be unaware that the compiler has
omitted them.
Pointer operations are more limited: we need only concern ourselves with
\texttt{inttoptr} and \texttt{ptrtoint}.
Here, we take performance penalties from generating excess instructions
and 128-bit integer operations.


\subsection{Address spaces}
\label{sec:impl-addrspace}

\subsubsection{Differences}
CHERI uses LLVM address space 200 for capabilities, and 0 for
non-capability pointers.
Supporting pure capability mode only requires use of address space 200.

However, the Rust compiler currently only generates code that uses
address space 0, the default address space in LLVM.
There is no support for multiple address spaces within or across
compilation units either.

\subsubsection{Choices and implications}
With pure capability mode, only one address space is required, thus the
latter problem is avoided.
In any case, there is no requirement to support legacy code and pointer
manipulation idioms in the scope of this work, thus little reason to
support the use of untagged pointers, and hence hybrid mode.

\subsubsection{Changes and limitations}
The main change was to make Rust aware of LLVM address spaces other than
0; it already specified this as the default.
Previous work to support Rust on the AVR platform already (in principle)
supported address space 1 for functions, so the changes required were to
determine the pointer width correctly from the LLVM data layout string,
and to ensure allocations were made to the correct address space, again
using the data layout string.

Based on the Rust codebase, it could be difficult but far from
impossible to support multiple address spaces in Rust.
A far bigger challenge, however, is to support code generation for
multiple pointer widths: it will therefore be impractical to consider
compiling Rust programs in hybrid mode unless this changes.


\subsection{Targeting CHERI}
\label{sec:impl-cheritarget}

\subsubsection{Differences}
The CHERI backend in LLVM is an extension of the big-endian 64-bit MIPS
architecture.
To target FreeBSD, it has the triple \cuf.

\subsubsection{Choices and implications}
In this case, the question is not so much what to change, but where to
change it.
For some changes, such as specifying the \texttt{cheri128} CPU (the
default being CHERI256), the Rust target specification files had
relevant fields.

In other cases, such as specifying the \texttt{purecap} ABI (as opposed
to the MIPS N64 ABI), no such mechanism was present in the compiler.
Here, either fields could be added to the target specification, or made
locally to their usage.


\subsubsection{Changes and limitations}
A target specification was added to the Rust compiler, specifying the
pure capability ABI, and the CHERI128 CPU.

Instead of adding more fields to the target specification structure and
code to parse these fields, I decided to make changes closest to where
the data would be used.
This minimised code changes.
However, this means that when targeting CHERI, pure capability mode
would always be used: I do not consider this an issue because the
CheriBSD can also run programs compiled with the MIPS N64 ABI, which can
also be compiled by the Rust compiler.
This choice should be revisited by if CHERI is to be made an official
target in the Rust compiler, as it is slightly unergonomic, even if the
ABI is not typically configured, and is not specified for other targets.


\section{Changes to core libraries}
\label{sec:impl-div-lib}

The changes to the compiler in \Cref{sec:impl-div-compiler} were not
sufficient to compile Rust programs.
Tests and benchmarks were not run against these libraries.

My attempts to compile the core libraries were cursory, only
investigating them as far as their functionality was required to run
test programs.
Therefore, it is possible that there are problems which I have not
noticed.
It is also possible that these changes might now be undone without
affecting the success of compilation: this is untested.

\subsection{\texttt{libcore}: formatting}
Three formatting methods were modified:

\begin{itemize}
    \item A string writing method, used for writing text to files or the
    console, was changed not to format strings before writing them.
    Thus variables cannot be printed.
    \item A pointer formatting method: this now prints a dummy string
    instead of an address.
    This was previously broken by invalid \texttt{ptrtoint} calls.
    \item Number formatting: a method to print numbers in arbitrary
    radices.
    Oddly, some of the pattern-matching code refused to compile.
\end{itemize}

These methods are not relevant to my evaluation as I have used the C
\texttt{printf} function for printing and formatting.


\subsection{\texttt{libcore}: UTF-8 validation}
The UTF-8 validation routine fails to compile into the core library: I
did not attempt to debug this.
I removed references to it.
Again, this is not relevant to my evaluation as I have not used Unicode.


\subsection{\texttt{libcore}: \texttt{memchr}}
The \texttt{memchr} implementation is similar in functionality to its C
version.
It performs pointer manipulation and bitwise operations, and did not
compile due to an iterator issue.


\subsection{\texttt{liballoc}: macro invocation}
Only a minor change was made: when returning an empty vector, to use
\texttt{Vec::new()} instead of the macro \texttt{vec![]}.
This was required because the macro was not in scope of the library as I
debugged it.


\section{Summary}

This chapter recorded and explained the strategy taken to evaluate Rust
on CHERI, which defines the scope of this project.
It documented techniques and processes which may be of use to future
implementers, as well as specific differences in the CHERI architecture
that lead to implementation conflicts with Rust.
For example, \Cref{sec:impl-width} explains implementation details of
the Rust compiler arising from the definition of index sizes, as well as
the choices made to resolve the issue for this project.

The strategy taken and compromises made set the stage for
\Cref{ch:eval}, where I evaluate the application of CHERI capabilities
to Rust, and Rust guarantees to CHERI.
They inform the evaluation and define its limitations.


\end{document}
