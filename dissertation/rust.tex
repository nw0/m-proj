\documentclass[dissertation.tex]{subfile}

\chapter{The Rust Programming Language}
\label{ch:rust}

This work refers to version 1.35 of the Rust
compiler.\footnote{Specifically, the nightly build dated March 22, 2019,
commit \texttt{2210e9a}. \\ Accessible at
\url{https://github.com/rust-lang/rust/commit/2210e9a6a99c4241d82e85ca71fd291d5ef91c7f}.
\\ Patches for CHERI compatibility at \url{https://github.com/CTSRD-CHERI/rust/}.}


\section{Overview}
This chapter gives an overview of Rust and the major differences from
comparable programming languages.

\wip{What else?}
Why Rust on CHERI? Rust vs Java, ML, etc.


\section{The Rust programming language}
Rust is billed as a fast and safe systems programming language.
It uses both compile-time and runtime checks to prevent overflows,
widely employing fat pointers for built-in data structures such as
\texttt{Vec}~\cite{blandy-orendorff}.

\wip{Probably a lot more to say}

\subsection{Optimisation opportunity}
With checks preventing out-of-bounds accesses, for instance, Rust might
initially appear to harbour large overheads.
Indeed, for code compiled in \emph{debug mode with assertions}, all
checks are performed, from checks on bounds to those on integer
overflows.
However, with optimisations enabled, many of these checks are elided:
for instance, array sizes are fixed at compile time, and most accesses
take the form of iteration, making a check redundant.
In this way, a number of checks are removed either by the Rust compiler
or LLVM.

\subsection{Unsafe Rust}
\label{sec:rust-unsafe}

\wip{Unsafety isn't unusual, many langs have escape hatches...}

Another area of optimisation is the use of \emph{Unsafe} Rust.
Unsafe Rust allows a few actions:

\begin{itemize}
    \item Dereferencing a raw pointer
    \item Calling unsafe functions
    \item Accessing or modifying a mutable static variable
    \item Implementing an unsafe trait~\cite{rust-unsafe-powers}
\end{itemize}

While these seem fairly innocuous (considering the large body of
existing C code), this is significant in the restrictions imposed on
safe Rust.
It's important to note that unsafety need not necessarily come from
the \texttt{unsafe} code block itself, but also from how the output of
such code is used.

Many unsafe optimisations in the Rust compiler take the form of
unchecked accesses: the suppression of unneeded or repeated checks, such
as slice accesses into objects of known size and layout.
Another example is arbitrarily indexing into a (Unicode) string, where
the chosen index is known to be a character boundary.

\subsection{Move semantics and borrowing}
