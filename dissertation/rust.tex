\documentclass[dissertation.tex]{subfiles}
\begin{document}

\chapter{The Rust Programming Language}
\label{ch:rust}

This work refers to version 1.35 of the Rust
compiler.\footnote{Specifically, the nightly build dated March 22, 2019,
commit \texttt{2210e9a}. \\ Accessible at
\url{https://github.com/rust-lang/rust/commit/2210e9a6a99c4241d82e85ca71fd291d5ef91c7f}.
\\ Patches for CHERI compatibility at \url{https://github.com/CTSRD-CHERI/rust/}.}


\section{Overview}
This chapter gives an overview of Rust and the major differences from
comparable programming languages.
It focusses on semantic differences, rather than usability differences,
although those also make a significant contribution to code
safety.\footnote{Usability differences include the lack of automatic
typecasting, the explicit use of integer widths (e.g.\ \texttt{i32}
rather than \texttt{int}, and \texttt{usize} for array indices), and the
built-in test harness.}

For any future effort in porting Rust to CHERI, attention \emph{must} be
paid to pointer width and indices (\Cref{sec:rust-usize}).
\emph{Currently, this effectively prevents Rust code from compiling
properly for CHERI.}

\wip{What else?}


\section{The Rust programming language}
Rust is billed as a fast and safe systems programming language.
It uses both compile-time and runtime checks to prevent overflows,
widely employing fat pointers for built-in data
structures~\cite{blandy-orendorff}.
For example, the built-in Vector (\texttt{Vec}) stores data on its
allocated capacity and actual length; a CHERI capability pointer to the
same vector would also track the allocated capacity as a boundary.
This redundancy suggests some room for optimisation.


\subsection{Unsafe Rust}
\label{sec:rust-unsafe}

To provide Rust with more power, \emph{Unsafe} Rust permits several
additional actions~\cite{rust-trpl-book}:

\begin{itemize}
    \item Dereferencing a raw pointer
    \item Calling unsafe functions
    \item Accessing or modifying a mutable static variable
    \item Implementing an unsafe trait
\end{itemize}

While these seem fairly innocuous (considering the large body of
existing C code), this is significant in the restrictions imposed on
Safe Rust.
It's important to note that unsafety need not necessarily come from
the \texttt{unsafe} code block itself, but from the handling of inputs
and outputs to the unsafe block.
\Cref{sec:eval-micro-repeat} shows a bug in a built-in data structure
caused by incorrect computation before an \texttt{unsafe} block, leading
to a buffer overflow.

As seen in \Cref{sec:rust-elision}, unsafe optimisations are used to
avoid redundant checks, whether they are duplicated or simply known to
be within bounds.
Unsafe code in the core library is also used for unchecked conversions,
or other unchecked indexing, like into a Unicode string at a known
character boundary.

Another example is to override the default allocation strategy
when initialising a vector with data: there is no need to fill the
allocated space with zeroes or poison values if it is guaranteed to be
overwritten before access.
Naturally, there is scope for programmer error here:
\Cref{sec:eval-micro-push} discusses an off-by-one error in
the standard library.


\subsection{Object ownership and borrow semantics}
\label{sec:rust-borrow}

\clar{So what?}

One of the guarantees Rust offers is that Safe Rust should never lead to
dangling pointers or memory leaks.
Yet it does not have a garbage collector, or expose memory management.
Instead, the compiler couples strict ownership analysis with borrow and
move semantics to determine the lifetime of an object.
As such, some programs which avoid use-after-free or other memory safety
bugs may be semantically invalid Rust.

\begin{figure}[ht]
    \lstinputlisting{listings/ownership-sigs.rs}
    \caption{
        \wip{improve example?}
        Function signatures for passing a \texttt{Vec} to a function.
        To prevent data races, only one code block can contain a mutable
        reference to a variable at a time.
    }
    \label{lst:rust-ownership}
\end{figure}

Instead of pointers, a function in Safe Rust will either take ownership
of an object, or borrow it mutably or immutably.
\Cref{lst:rust-ownership} shows some function signatures for
moving ownership or borrowing.
In contrast to C, the emphasis is not on access, but the nature of the
access.\footnote{In Rust, references represent possession of an object,
rather than the ability to access it.}
As such, pointers cannot be casually duplicated\footnote{In Safe Rust.}
to cause data races or temporal unsoundness generally.

Ownership does not guarantee temporal safety on its own.
However, it makes ownership explicit and provenance clear, preventing
bugs such as use-after-free.
\wip{unclear}

\subsubsection{Caveat on memory leaks}
An exception to the memory leak guarantee is the exposure of
\texttt{mem::forget} in the core library as a safe function.
This allows a programmer to end an object's lifetime to end without
deallocating it, used to avoid double-frees in circular data structures.
This can create a leak, as can a knot-tied reference-counted data
structure, though knot-tying is made difficult in light of the single
mutable reference rule.
\Cref{sec:eval-rust-use-after-free} gives an example of a memory leak in
Safe Rust.


\subsection{Optimisation by eliding checks}
\label{sec:rust-elision}

For spatial safety, Rust implements runtime bounds checks to
prevent out-of-bounds accesses.
This may appear to harbour large overheads, although with optimisations
many checks are elided.
\Cref{lst:rust-iter-unchecked} shows examples of bounds check elision in
generated code, based on iterators and detecting when the length is
checked.
This detection is basic, as we see in \Cref{lst:rust-iter-checked}: it
is more effective to use built-in iterators in general.

Some common situations in which bounds checks are avoided:

\begin{description}
    \item[Built-in binary search] Checks avoided without unsafe code by
    providing more information to compiler: uses slices instead of
    indices~\cite{rust-pr-binarysearch}.
    \item[Built-in slice equality] Similar to \textbf{binary search},
    providing more information to the compiler makes a different
    iteration strategy faster without unsafe dereferencing~\cite{rust-pr-slice-partialeq}.
    \item[Built-in iterators] The default iterator applying to arrays
    (\Cref{lst:rust-iter-unchecked}) uses the unsafe
    \texttt{get\_unchecked} method\footnote{
    This adds an offset to the array's base pointer and dereferences it
    directly.
    Other \texttt{unchecked} methods exist, e.g.\ in string
    conversions.
    } implemented for the \texttt{SliceIndex} trait.
    In theory, this \emph{might} be susceptible to an off-by-one error,
    but such a mistake is easily discovered in unit tests.
    \Cref{sec:rust-unsafe} covers how unsafe code is used for optimisation.
    \item[Image processing] Dr\"oge demonstrates different optimisation
    strategies without resorting to Unsafe Rust, while still avoiding
    bounds checks and other unnecessary operations~\cite{droge-opt}.
    This is achieved through strategic assertions and optimal use of
    iterators and other built-ins.
    Note that similar optimisation through assertions may be fragile or
    made obsolete through updated code generation, and that optimal use
    of built-ins may require knowledge of the underlying (unsafe)
    implementations.
\end{description}

Therefore when writing idiomatic Rust, bounds checks mainly apply to
random indexing, as seen in \Cref{lst:bg-rust-index}.
They also apply when Rust's intrinsic indexing is used, through the
\texttt{Index} and \texttt{IndexMut} traits.
For example, \Cref{lst:rust-vec-index} shows that \texttt{Vec} uses
this intrinsic, and therefore has bounds checks.
Consequently, the overall performance impact of bounds checks is small.

\begin{figure}[ht]
    \lstinputlisting{listings/vec-index-trait.rs}
    \caption{
        Like built-in arrays, Rust's \texttt{Vec} uses intrinsic
        indexing.
        The same bounds checks therefore apply to random indexing into a
        \texttt{Vec}, or any structure built on one.
    }
    \label{lst:rust-vec-index}
\end{figure}

\begin{figure}[ht]
    \lstinputlisting{listings/sums-iter.rs}
    \lstinputlisting[language={[mips64]Assembler}]{listings/array-itersum.s}
    \caption{
        Four ways to sum an array in Rust: the first three identically
        generate the shown MIPS assembler, and could be considered
        idiomatic Rust.
        The fourth generates nearly the same code: it uses an index of 1
        instead of 8, allowing the omission of lines~3--4!
        It also omits lines~12--13, which appear to be redundant.
        Note that no bounds checks are present.
        Optimised code generated; comments added for ease of reading.
    }
    \label{lst:rust-iter-unchecked}
\end{figure}

\begin{figure}[htp]
    \lstinputlisting{listings/sum-checked.rs}
    \lstinputlisting[language={[mips64]Assembler}]{listings/array-checkedsum.s}
    \caption{
        How not to sum an array in Rust.
        Observe that \texttt{sum\_checked} is nearly identical to
        \texttt{sum\_loop} in \Cref{lst:rust-iter-unchecked}, save for
        the comparison on line~5.
        The compiler fails to detect that our accesses are safe due to
        the small manipulation of the variable storing the length.
        Thus an additional branch instruction appears on line~14.
        Optimised code generated;
        \texttt{panic\_bounds\_check} string substituted in place of
        mangled version.
    }
    \label{lst:rust-iter-checked}
\end{figure}

\subsubsection{Another approach to removing bounds checks}
Finally, there is an experimental effort~\cite{rust-crate-indexing} to
remove array bounds checks using dependent types, using the principles
described by Xi and Pfenning~\cite{xi-pfenning-dependent} and applied to
Rust by Beingessner~\cite{beingessner}.
Note that this could add a significant compile-time
overhead, as it necessarily utilises type-, lifetime, and ownership
checkers, which Rust runs before optimisations: this results in
heavy computation to remove checks that are usually avoided anyway.


\subsection{Definitions of pointers and indices}
\label{sec:rust-usize}

In Rust, a \texttt{usize} is defined to be a pointer-sized integer,
conventionally equivalent to C's \texttt{uintptr\_t}, which \emph{may
not be equivalent to \texttt{size\_t}}~\cite{rust-rfc-isize}.
The width of a \texttt{usize} derives from the data layout's pointer
width, 128 bits for a CHERI capability.
Crucially, \texttt{usize} is used as the index size in Rust, for
indexing into arrays, structs, and all objects, and is hence passed to
LLVM instructions such as \texttt{getelementptr}.

\subsubsection{Implications}
Having a 128-bit \texttt{usize} means that the Rust compiler instructs
LLVM to generate 128-bit indexed versions of intrinsics such as
\texttt{memcpy}, in addition to indexing.
However, not all bit widths for the \texttt{memcpy} intrinsic are
supported by all targets~\cite{llvm-langref}, and CHERI supports a bit
width of 64 but not 128.

As it stands, this is an implementation detail which is the consequence
of the language semantics.
Nevertheless, to target CHERI properly, Rust must either support pointer
widths larger than the index size, or use pointer widths differently in
code generation.
\Cref{sec:impl-width} explores the implementation in more detail.


\section{Summary}
\todo{summarise\ldots}

\end{document}
