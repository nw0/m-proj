\documentclass[11pt]{article}
\usepackage{a4wide,parskip,times}

\usepackage[utf8]{inputenc}
\usepackage[T1]{fontenc}
\usepackage{inconsolata}
\usepackage{color}
\usepackage{listings}
\lstdefinelanguage
   [x64]{Assembler}     % add a "x64" dialect of Assembler
   [x86masm]{Assembler} % based on the "x86masm" dialect
   % with these extra keywords:
   {morekeywords={CDQE,CQO,CMPSQ,CMPXCHG16B,JRCXZ,LODSQ,MOVSXD, %
                  POPFQ,PUSHFQ,SCASQ,STOSQ,IRETQ,RDTSCP,SWAPGS, %
                  rax,rdx,rcx,rbx,rsi,rdi,rsp,rbp, %
                  r8,r8d,r8w,r8b,r9,r9d,r9w,r9b, %
                  r10,r10d,r10w,r10b,r11,r11d,r11w,r11b, %
                  r12,r12d,r12w,r12b,r13,r13d,r13w,r13b, %
                  r14,r14d,r14w,r14b,r15,r15d,r15w,r15b}} % etc.

\lstset{language={[x64]Assembler},frame=single,basicstyle=\ttfamily}
\usepackage{microtype}

\begin{document}

\centerline{\Large Optimising Rust memory safety with CHERI capabilities}
\vspace{2em}
\centerline{\Large \emph{An MPhil project proposal}}
\vspace{2em}
\centerline{\large N. W. S. Sim (\emph{nwss2}), Wolfson College}
\vspace{1em}
\centerline{\large Project Supervisor: Prof Simon Moore}
\vspace{1em}

\begin{abstract}
\textsl{
Memory-safe languages such as Rust are a safer alternative to systems programming in C, as they prevent traditional paths to security exploits, such as buffer overflows and unauthorised reads.
However, runtime checks such as those on array overrun or invalid type/trait coercion to maintain safety incur performance penalties.
The CHERI architecture provides \emph{capabilities} in hardware, which eliminate the possibility of accessing or modifying unauthorised regions of memory while incurring minimal overheads.
We therefore propose to extend Rust to use capabilities, potentially providing stronger safety guarantees while reducing runtime overheads, where safety guarantees would have been absent before.
(98 words)} 
\end{abstract}

\section{Introduction, approach and outcomes (402 words)}

\textsl{Provide an introduction to your project or essay. In particular, try to
  motivate the work and explain the relevant context (general
  background, as well as sufficient detail about any related
  work).}

\textsl{What's the basic idea and approach? What are you thinking of 
doing, and how is it going to solve the problem (or need) you've 
identified. What are you going to ``produce''? 
A project will typically produce one (or perhaps more) of the following:
a piece of software, an evaluation of a published result, a proof, or
the design (and perhaps the construction of) a new piece of hardware. An
essay will typically either review and critique a particular area of the
academic literature, or evaluate a published result or proof. Try to 
name the specific things (and describe them) in this part of the 
proposal -- this will allow you to refer to them in the next.} 

Many security vulnerabilities are bugs arising from the lack of memory safety in C; a traditional exploit is the buffer overflow.
The C language specification includes a number of situations that lead to \emph{undefined behaviour}: akin to a deduction of `falsity' in logic, after these points the state of the program might be arbitrary.
Undefined behaviour allows compilers to assume that certain conditions never hold, enabling them to make powerful (but dangerous) optimisations.

Consequently, crucial checks can be optimised away and invariants violated by compilers, resulting in executables vulnerable to memory attacks~\cite{simon2018wygiwyc}.
Many systems and compiler programmers frequently underestimate the pervasiveness of undefined behaviour in code that appears to function correctly, and gloss over the subtleties of the C standard~\cite{memarian2016cdepths};
these represent future opportunities for vulnerabilities to arise when code is optimised away by compiler transformations~\cite{wang2013towards}.

The CHERI project provides \emph{capabilities}, which stops unauthorised access or modification to data by erroneous and malicious code~\cite{cheri-v6}.
In the CHERI architecture, capabilities replace pointers and internal addresses to prevent unauthorised use; capabilities restrict usage by storing permissions alongside the address data while making these unforgeable, as they can only be manipulated by the hardware.

Rust is a \emph{memory-safe} language designed for systems programming.
It includes runtime checks that prevent many of these attacks, but these safety measures incur a performance overhead.
Figure~\ref{lst:array-check} shows an example of a runtime check.
The semantics of Rust allow us to precisely determine which operations are valid, as they provide detailed insight into data ownership and lifetimes.
We will modify some of these checks to use capabilities instead, providing a performance speedup on the CHERI architecture.

\begin{figure}[ht]
\begin{lstlisting}
example::iterate:
  sub     rsp, 24
  mov     eax, 100                  ; access array index 100
  mov     ecx, eax
  cmp     rcx, rsi
  setb    dl
  test    dl, 1
  mov     qword ptr [rsp + 16], rdi
  mov     qword ptr [rsp + 8], rsi
  jne     .LBB0_1
  jmp     .LBB0_2
.LBB0_1:
  mov     rax, qword ptr [rsp + 16]
  mov     eax, dword ptr [rax + 400]
  add     rsp, 24
  ret
.LBB0_2:
  lea     rdi, [rip + .L__unnamed_1]
  mov     esi, 100
  mov     rdx, qword ptr [rsp + 8]
  call    core::panicking::panic_bounds_check@PLT
  ud2
\end{lstlisting}
\caption{Compiled array access in function: a bounds check manifests as a branch instruction on line 10. This adds complexity to program execution paths; instead of this check, we can simply trap a hardware exception in CHERI.}
\label{lst:array-check}
\end{figure}

Intended contributions:
\begin{itemize}
  \item Functional Rust (cross-)compiler to CHERI which modifies redundant runtime checks, such as array bounds checks.
  \item A set of microbenchmarks (test cases) consisting of proof-of-concept exploits which are stopped by both capabilities and checks; these represent the checks that may be modified.
  \item Performance evaluation on the resultant Rust compiler, with a mixture of test examples, utility library code (such as sorting algorithms), and relevant systems code.
  \item A \emph{non}-contribution: exhaustive removal of redundant checks: the project is intended to illustrate and quantify the benefit of using CHERI capabilities to optimise a safe language, rather than a full implementation of an optimised memory-safe language. It may be acceptable to break some language features that do not contradict the resultant memory model.
\end{itemize}

\section{Workplan (426 words)}

\begin{description}
  \item [Dec 10--23] Initial project work: set up of environment, toolchain; finesse a working process for compiling Rust programs to CHERI.
  Survey of literature for capabilities and their implications on memory-safe languages with runtime checks and Rust specifically.
  \item [Dec 24--Jan 6] Devise tests and microbenchmarks as mentioned in intended contributions: proof-of-concept exploits that could be stopped by either capabilities or runtime checks; these represent the checks that may be optimised out.
  \item [Jan 7--Jan 20] Identify relevant checks (code in the Rust compiler) that may be modified, attempt to do so and consider if changes at other layers of the compiler need to be made; derive tests that need to be made to ensure that the compiler is still operable and that other runtime checks or security properties are not compromised: considerable familiarity with the Rust documentation desired before this point.
  \item [Jan 21--Feb 3] Continue modifying redundant checks for capabilities, continually testing that this works.
  Provide Rust code (as a demonstration) that the checks are not present in the LLVM intermediate representation, as required.
  Additional (fresh) review of related work.
  \item [Feb 4--17]  Finalise the modification of checks and functional testing.
  \item [Feb 18--Mar 3] Experimental performance testing: simple cases including simple array checks, library vector implementations to determine factors that affect performance.
  Consider and design experiments for the desired factors: decide how `performance' should be quantified and produce relevant test methodology.
  \item [Mar 4--17] Initial performance testing with straightforward but realistic and relevant systems programming applications.
  Refine test methodology as needed: the objective is to gather simple and relevant results but prepare the methodology for more detailed testing in the next stage.
  At this point, tests should be simple to re-run after tweaking methods.
  \item [Mar 18--31] More intricate performance testing: traits and generics, quicksort etc.
  These are particularly relevant for Rust to be useful as a general programming language on CHERI.
  \item [Apr 1--14] Examine implications of capability-aware \texttt{memcpy} and other instances of calling into C for CHERI-Rust.
  On other platforms, when Rust calls into C, the returned data is checked for safety: I want to consider if these checks can too be modified if they are made redundant by CHERI capabilities.
  \item [Apr 15--28] Examine results and performance visualisation; final literature review (to continue into dissertation writing).
  Consolidation of modifications made and relevant test cases, examples, or proofs.
  Determine if additional results are required or helpful, and perform them.
  \item [Apr 29--May 13] Dissertation writing: continuing review to ensure relevance, writing up of results, interpretation, and limitations.
  \item [May 13--26] Dissertation writing: reconsider context and motivation, contemplate future work.
  \item [May 27--Jun 9] Contingency.
\end{description}

\bibliography{../project}
\bibliographystyle{plain}

\newpage
\appendix

\end{document}
